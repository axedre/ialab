In quest'ultima sezione illustreremo alcune features interessanti, usate per questo progetto, che esulano dal programma d'esame.
\section{Repository Git} \label{sec:repo}
Ai fini di un adeguato meccanismo di versioning per la collaborazione tra gli studenti autori del progetto, si è scelto di utilizzare git come meccanismo di versioning. Questo ha comportato un sistema affidabile per il tracciamento di versioni e modifiche, con la possibilità di tenere traccia, ed eventualmente facendo delle operazioni di revert, degli sviluppi incrementali dei vari progetti. È stato reso disponibile il repository su github\footnote{il repository è disponibile su github all'indirizzo https://github.com/axedre/ialab}.

\section{Sistema di collaborazione} \label{sec:collaborazione}
Un altro strumento particolarmente utile alla collaborazione e alla coordinazione del lavoro è stato l'utilizzo del tool gratuito Zoho Projects\footnote{https://www.zoho.com/projects/}, che ci ha consentito di dividerci i compiti, tener traccia dei compiti da svolgere, dei problemi da risolvere e di avere una message board ordinata che consentisse di tener traccia del lavoro svolto e da svolgere.
È stato inoltre possibile fissare delle date di scadenza sui vari compiti in maniera da organizzare in maniera chiara, ordinata e funzionale il flusso di lavoro.

\section{Migliorie nell'interfaccia grafica} \label{sec:grafica}
Importanti migliorie nell'interfaccia grafica java sono apprezzabili. La principale è stata l'integrazione dei punteggi sulla mappa, in maniera da rendere più intuitivo il comportamento dell'UAV. A essa è stato aggiunto anche il disegno di un bersaglio per rappresentare la cella verso cui l'UAV è diretto.
Sempre ai fini di una più chiara situazione sulla mappa vengono colorate in verde le celle su cui è già stata fatta una inform mentre sono illustrate in rosso quelle su cui deve essere ancora fatta.

\section{Generatore di mappe} \label{sec:generatore}
Infine, mediante l'utilizzo delle tecnologie AngularJS e Node.js, è stato costruito un generatore di mappe\footnote{disponibile all'indirizzo http://mmg-axedre.rhcloud.com/} che consente di generare, in maniera grafica, i file di configurazione .clp delle mappe pre e post allagamento, interamente personalizzabili in ogni aspetto. Questo ha reso più snelli e veloci i test delle varie strategie con ogni tipologia di mappa.

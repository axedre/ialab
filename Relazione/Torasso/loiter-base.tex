La strategia chiamata Loiter Monitoring con punteggi relativi è una variante della strategia illustrata nel capitolo \ref{cap:rel}. La principale variante consiste nel modulo inform, in cui viene prevista un'azione di Loiter Monitoring se l'UAV si trova su una cella di tipo urban (e ha sufficiente tempo rimanente), come spiegato nel modulo \ref{sec:loiter-base-inform}.

\section{Modulo: Controllo del tempo rimanente} \label{sec:loiter-base-tempo}
Questo modulo è molto simile a quelli corrispondenti nelle altre strategia, vi è tuttavia un'aggiunta significativa giustificata dalla variazione della strategia di inform. Siccome l'operatione di Loiter Monitoring è piuttosto costosa, nel momento in cui il tempo comincia a scarseggiare, ovvero quando scende sotto la soglia di $ \frac{tempo}{numero-celle} \cdot 50 $\footnote{l'idea è che il costo di un'operazione loiter è 50: si stabilisce che una volta che si è scesi sotto una soglia proporzionale al costo dell'operazione di loiter e al tempo a disposizione, inversamente proporzionale al numero di celle potenzialmente informabili, allora si continua effettuando solo delle inform semplici}, si asserisce un fatto di tipo \texttt{no-loiter} che impedisca di effettuare quest'onerosa operazione. L'idea nasce dal fatto che è desiderabile fare più inform, anche se meno precise, e lasciare poche celle non analizzate, piuttosto che fare poche inform precise lasciando, però, molte celle non analizzate.

\section{Inform delle celle visibili} \label{sec:loiter-base-inform}
A ogni esecuzione di un movimento dell'UAV viene dato il focus a un modulo di inform. Questo modulo ha la responsabilità di effettuare azioni di tipo \texttt{inform} di celle sulle quali non è ancora stata effettuata tale azione secondo i criteri della strategia. Viene effettuata una \texttt{inform} base per ogni nuova cella avvistata, in quanto il costo dell'operazione di \texttt{inform} è basso. Quando, però, la nuova cella avvistata è di tipo \texttt{urban}, e si ha una percezione di tipo water su di essa, allora, se non è stato asserito il fatto \texttt{no-loiter}, si procede con una azione di loiter monitoring sulla cella in questione, quindi si predispone, attraverso un fatto di tipo \texttt{inform-act}, una inform precisa (\texttt{severe-flood} o \texttt{initial-flood}) che verrà poi effettuata dalla regola \texttt{exec-inform} nel modulo \texttt{AGENT}.

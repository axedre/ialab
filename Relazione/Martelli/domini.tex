In questa sezione verranno illustrati i domini presi in esame per gli esercizi. Vediamo subito quali sono:
\section{Dominio dei Cammini (10x10)} \label{sec:cammini:10x10}
Il dominio dei cammini consiste in un inseme di posizioni (100, disposte con coordinate di un quadrato 10x10), tra cui una iniziale e una finale, che possono essere libere oppure occupate. Alla luce di ciò si deve cercare, mediante una ricerca in profondità, una sequenza di azioni che portino dalla posizione iniziale a quella finale, passando da posizioni definite (entro i confini della mappa insomma) ed evitando le posizioni occupate, precedentemente definite nel dominio.
\section{Dominio dei Cammini (20x20)} \label{sec:cammini:20x20}
Il dominio dei cammini consiste in un inseme di posizioni (400, disposte con coordinate di un quadrato 20x20), tra cui una iniziale e una finale, che possono essere libere oppure occupate. Alla luce di ciò si deve cercare, mediante una ricerca in profondità, una sequenza di azioni che portino dalla posizione iniziale a quella finale, passando da posizioni definite (entro i confini della mappa insomma) ed evitando le posizioni occupate, precedentemente definite nel dominio.
\section{Dominio del mondo dei blocchi} \label{sec:blocchi}
La descrizione del dominio del mondo dei blocchi è nota in letteratura. Consiste nell'avere a disposizione dei blocchi che possono essere inpilati uno sopra l'altro: l'ordine risultante sarà l'ordine inverso delle operazioni di impilamento effettuate\footnote{ad esempio, se sopra al blocco A si appoggia il blocco B, quindi sopra al blocco B si appoggia il blocco C, l'ordine risultante sarà: il blocco C è sopra al blocco B, il quale è sopra al blocco A, che, come si vedrà, appoggia sul tavolo.}. I blocchi possono essere sopra a un altro blocco oppure sul tavolo. Sul tavolo ci possono essere infiniti blocchi, mentre su un blocco può esserci, immediatamente, un solo blocco.

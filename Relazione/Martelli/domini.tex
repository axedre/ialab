In questa sezione verranno illustrati i domini presi in esame per gli esercizi. Vediamo subito quali sono:

\section{Dominio dei Cammini (10x10)} \label{sec:cammini:10x10}
Il dominio dei cammini consiste in un inseme di posizioni (100, disposte con coordinate di un quadrato 10x10), tra cui una iniziale e una finale, che possono essere libere oppure occupate. Alla luce di ciò si deve cercare una sequenza di azioni che portino dalla posizione iniziale a quella finale, passando da posizioni definite (entro i confini della mappa insomma) ed evitando le posizioni occupate, precedentemente definite nel dominio.

\section{Dominio dei Cammini (20x20)} \label{sec:cammini:20x20}
Il dominio dei cammini consiste in un inseme di posizioni (400, disposte con coordinate di un quadrato 20x20), tra cui una iniziale e una finale, che possono essere libere oppure occupate. Alla luce di ciò si deve cercare una sequenza di azioni che portino dalla posizione iniziale a quella finale, passando da posizioni definite (entro i confini della mappa insomma) ed evitando le posizioni occupate, precedentemente definite nel dominio.

\section{Dominio del mondo dei blocchi} \label{sec:blocchi}
La descrizione del dominio del mondo dei blocchi è nota in letteratura. Consiste nell'avere a disposizione dei blocchi che possono essere inpilati uno sopra l'altro. Ogni blocco può essere sopra a un altro blocco oppure sul tavolo. Sul tavolo ci possono essere infiniti blocchi, mentre su un blocco può esserci, immediatamente, un solo blocco. Viene inoltre specificato, tramite il fatto \texttt{clear}, quando sopra a un blocco non ce n'è un altro.
Vi sono, inoltre, due configurazioni proposte per questo dominio. La prima è meno complessa in quanto presenta pochi blocchi. La seconda configurazione è più intricata, con più blocchi rispetto alla prima e più mosse necessarie per arrivare al goal.

\section{Dominio della Metropolitana di Londra} \label{sec:tube}
Come si evince dal nome di questo dominio, in questa sede viene illustrata parte della metropolitana di Londra. Vi è una suddivisione in Linee, caratterizzate da una lista di fermate (stazioni della metropolitana) e da due possibili direzioni. Ogni fermata è caratterizzata da un nome e da due coordinate indicanti la dislocazione rispetto alla stazione di London Bridge, avente infatti coordinate \texttt{(0, 0)}.
Le azioni possibili sono quella di salire (\texttt{Sali(Linea, Direzione)}, possibile quando ci si trova in una stazione della Linea su cui si desidera salire. È poi possibile scendere (\texttt{Scendi(Stazione)}, quando si è su un treno di una linea in cui è presente la Stazione indicata. Infine vi è l'azione di Vai (\texttt{Vai(Linea, Direzione, StazionePartenza, StazioneArrivo}), che indica, una volta che si è su un treno, lo spostamento da una stazione all'altra (consecutive) su una stessa linea.

\documentclass[a4paper,12pt, twoside]{report}
 
\usepackage[italian]{babel}
\usepackage{amsmath}
% Adatta LaTeX alle convenzioni tipografiche italiane,
% e ridefinisce alcuni titoli in italiano, come "Capitolo" al posto di "Chapter",
% se il vostro documento è in italiano

\usepackage[utf8]{inputenc} % Consente l'uso caratteri accentati italiani
\usepackage{fancyhdr, color}

\usepackage{graphicx} % Consente l'uso di immagini
\usepackage{caption}
\usepackage{subcaption}

% Per disegnare flowchart
\usepackage{tikz}
\usetikzlibrary{shapes.geometric, arrows}
\tikzstyle{startstop} = [rectangle, rounded corners, minimum width=3cm, text centered, draw=black, fill=red!30]
\tikzstyle{process} = [rectangle, minimum width=3cm, text centered, draw=black, fill=orange!30]
\tikzstyle{decision} = [diamond, minimum width=3cm, text centered, draw=black, fill=green!30]
\tikzstyle{arrow} = [thick,->,>=stealth]

% Per includere porzioni di codice
\usepackage{listings}
\lstset{
language=Prolog,
basicstyle=\scriptsize\sffamily,
numbers=left,
numberstyle=\scriptsize,
frame=tb,
columns=fullflexible,
showstringspaces=false
}

\makeindex
\linespread{1}

\pagestyle{fancy}
\renewcommand{\chaptermark}[1]{\markboth{\MakeUppercase{\chaptername}\  \thechapter.\ #1}{}}
\renewcommand{\sectionmark}[1]{\markright{\thesection.\ \ #1}{}}
\fancyhead{}
\fancyhead[LE]{\slshape \leftmark}
\fancyhead[RO]{\slshape \rightmark}
\fancyfoot[EC,OC]{\thepage}
\renewcommand{\headrulewidth}{0.4pt}
 
\title{Relazione esercizi PROLOG e CLINGO}
\author{Maurizio Dominici, Alessandro Serra, Andrea Aloi}
\date{\today}
 
\begin{document}
\pagenumbering{Roman}
\maketitle
 
\tableofcontents
\newpage

\pagenumbering{arabic}
\part{PROLOG}
\chapter{Domini}
In questa sezione verranno illustrati i domini presi in esame per gli esercizi. Vediamo subito quali sono:
\section{Dominio dei Cammini (10x10)} \label{sec:cammini:10x10}
Il dominio dei cammini consiste in un inseme di posizioni (100, disposte con coordinate di un quadrato 10x10), tra cui una iniziale e una finale, che possono essere libere oppure occupate. Alla luce di ciò si deve cercare, mediante una ricerca in profondità, una sequenza di azioni che portino dalla posizione iniziale a quella finale, passando da posizioni definite (entro i confini della mappa insomma) ed evitando le posizioni occupate, precedentemente definite nel dominio.
\section{Dominio dei Cammini (20x20)} \label{sec:cammini:20x20}
Il dominio dei cammini consiste in un inseme di posizioni (400, disposte con coordinate di un quadrato 20x20), tra cui una iniziale e una finale, che possono essere libere oppure occupate. Alla luce di ciò si deve cercare, mediante una ricerca in profondità, una sequenza di azioni che portino dalla posizione iniziale a quella finale, passando da posizioni definite (entro i confini della mappa insomma) ed evitando le posizioni occupate, precedentemente definite nel dominio.
\section{Dominio del mondo dei blocchi} \label{sec:blocchi}
La descrizione del dominio del mondo dei blocchi è nota in letteratura. Consiste nell'avere a disposizione dei blocchi che possono essere inpilati uno sopra l'altro: l'ordine risultante sarà l'ordine inverso delle operazioni di impilamento effettuate\footnote{ad esempio, se sopra al blocco A si appoggia il blocco B, quindi sopra al blocco B si appoggia il blocco C, l'ordine risultante sarà: il blocco C è sopra al blocco B, il quale è sopra al blocco A, che, come si vedrà, appoggia sul tavolo.}. I blocchi possono essere sopra a un altro blocco oppure sul tavolo. Sul tavolo ci possono essere infiniti blocchi, mentre su un blocco può esserci, immediatamente, un solo blocco.


\chapter{Ricerche in profondità}
In questa parte verranno illustrati gli esercizi inerenti la ricerca in profondità sui vari domini.

\section{Dominio dei Cammini (10x10)}

\subsection{Ricerca in profondità semplice} \label{sub:10-prof-semplice}
La prima ricerca che prendiamo in esame è la così detta ricerca semplice, ovvero una ricerca senza controlli di cicli. Come ci si aspetterebbe, a causa dell'assenza di controlli sugli stati già esplorati, una volta lanciato l'interprete prolog con la ricerca semplice viene restituito \texttt{Out of local stack}. Questo è dovuto, in effetti, a una ricursione infinita in quanto i cammini possibili prevedono cicli. L'interprete espande sempre la chiamata ricursiva più "profonda" e, la presenza di cicli nei cammini, non garantisce la terminazione.

\subsection{Ricerca in profondità con controllo di cicli} \label{sub:10-prof-cc}
Per ovviare a questo problema si è implementata la ricerca in profondità con controllo di cicli. L'idea è quella di utilizzare un insieme indicante le posizioni precedentemente visitate, in maniera da "potare" i rami decisionali generanti cicli. L'insieme è chiamato \texttt{visitati} e, a ogni passo di ricorsione, gli viene aggiunta la posizione attuale.

\subsection{Ricerca in profondità limitata} \label{sub:10-prof-limitata}
La ricerca in profondità limitata consiste nel definire un limite di profondità entro il quale la ricerca si deve fermare. Se è possibile dimostrare l'esistenza di un percorso lungo al più quanto il limite fornito viene restituita la sequenza risultato. Viceversa vi sarà un fallimento.
L'implementazione usata per questa strategia consiste nel passaggio del limite a ogni chiamata ricorsiva. Esso viene decrementato di uno al passaggio alla successiva chiamata ricorsiva e, ogni volta, viene controllato che questo limite sia maggiore di zero. Risulta chiaro che una volta arrivati alla base della ricorsione si è a un livello di profondità pari a quello inizialmente definito. Come detto: se il goal viene raggiunto entro il livello stabilito si ha la soluzione, altrimenti, arrivati alla base della ricorsione, si ha un fallimento.

\subsection{Ricerca in profondità ad approfondimento iterativo} \label{sub:10-prof-it}
Per questa versione viene riutilizzata la precedente ricerca (\ref{sub:10-prof-limitata}) impostando il limite inizialmente a 1 (quindi si avrà successo solo nel caso in cui posizione iniziale e finale coincidano). A ogni iterazione viene incrementato il limite di un'unità e nuovamente esplorato il cammino con la la ricerca in profondità limitata. Il principale vantaggio di questo approccio è la sicurezza di trovare una soluzione minima, in quanto i passi hanno tutti ugual costo (non ci sono cammini pesati) e in quanto il procedimento può essere visto (seppure il comportamento sia diverso) in maniera simile a una ricerca in ampiezza. Questo non è del tutto vero: infatti ogni volta si esplorerà in profondità il ramo in questione e, solo successivamente, si esploreranno, sempre in profondità, i nodi di un altro ramo. Tuttavia, considerando il fatto che si esplorerà sempre prima un cammino più corto di uno lungo, la prima soluzione trovata, come per la ricerca in ampiezza, avrà la proprietà desiderabile di essere anche la soluzione minima. In effetti questa strategia combina i vantaggi della ricerca in profondità con quelli della ricerca in ampiezza.

%-----------------------------------------------------------------------------------------------------------
\section{Dominio dei Cammini (20x20)}

\subsection{Ricerca in profondità semplice} \label{sub:20-prof-semplice}
Anche in questo caso, come per il dominio illustrato nella sezione \ref{sec:cammini:10x10}, verrà restituito l'errore \texttt{Out of local stack}. I motivi sono i medesimi illustrati nella sezione \ref{sub:10-prof-semplice}.

\subsection{Ricerca in profondità con controllo di cicli} \label{sub:20-prof-cc}
Come fatto precedentemente, nella sezione \ref{sub:10-prof-cc}, per risolvere il problema \ref{sub:20-prof-semplice}, si è utilizzato un controllo di cicli. Non si ripete il procedimento in questa sede in quanto sufficientemente illustrato nella sezione \ref{sub:10-prof-cc}.

\subsection{Ricerca in profondità limitata} \label{sub:20-prof-limitata}
Sebbene il funzionamento sia del tutto simile a quanto illustrato nella sezione \ref{sub:10-prof-limitata}, la maggiore profondità del dominio implica un numero sensibilmente maggiore delle possibilità di esplorazione in profondità. Questo si ripercuote in maniera piuttosto evidente sui tempi di esecuzione, di gran lunga superiori a quelli del dominio di dimensioni più piccole. È un fatto sicuramente atteso, dovuto all'enorme differenza di numero di possibili percorsi di una data lunghezza rispetto al dominio \ref{sec:cammini:10x10}.

\subsection{Ricerca in profondità ad approfondimento iterativo} \label{sub:20-prof-it}
Le maggiori conseguenze enfatizzate nella sezione \ref{sub:20-prof-limitata} si riscontrano con questa strategia. In effetti, assumendo che la soluzione minima sia lunghezza \texttt{l}, sarà necessario esplorare prima tutti i possibili cammini di tutte le lunghezza che vanno da \texttt{1} a \texttt{l-1} e, constatando quanto ci si possa impiegare a trovare tutti i cammini di una data lunghezza a ogni livello, il tempo di attesa è molto lungo. Le stampe inserite a ogni nuova chiamata con lunghezza maggiore sono indizio di ciò (più il numero della profondità massima aumenta e più, ovviamente, queste stampe si fanno rarefatte).

%-----------------------------------------------------------------------------------------------------------
\section{Dominio del mondo dei blocchi - prima configurazione}

Su questo dominio non ci sono significative differenze rispetto al Dominio dei cammini, conseguentemente non spenderemo ulteriori commenti

%-----------------------------------------------------------------------------------------------------------
\section{Dominio del mondo dei blocchi - seconda configurazione}

Più interessante è la seconda configurazione che, essendo molto più complessa della prima, causa un incremento importante dei tempi necessari a trovare una soluzione. Tralasciando la Ricerca in profondità semplice, che anche qui, a causa di cicli, porta a un errore \texttt{Out of local stack}.

\subsection{Ricerca in profondità con controllo di cicli} \label{sub:blocchi-prof-cc}
La ricerca in profondità con controllo di cicli risulta, a causa della natura del dominio, molto lunga. Il risultato trovato è ben distante da una soluzione ottimale, questo è dovuto al fatto che, in effetti, l'algoritmo si occupa solamente di evitare i possibili cicli, tuttavia, l'insieme delle mosse possibili, molto vasto: conseguentemente la soluzione include diverse mosse non ottimali che inficiano sulle prestazioni temporali.

\subsection{Ricerca in profondità ad approfondimento iterativo}
Anche qui vengono spese ulteriori parole sulla Ricerca in profondità limitata e si passa subito alla Ricerca in profondità ad approfondimento iterativo. I problemi temporali incontrati in \ref{sub:blocchi-prof-cc} si ripercuotono, ovviamente anche qui; tuttavia questo dominio è particolarmente interessante in quanto permette di evidenziare una proprietà fondamentale dell'iterative deepening: nonostante il tempo impiegato sia paragonabile alla soluzione proposta in \ref{sub:blocchi-prof-cc}, con l'approfondimento iterativo viene trovata una soluzione ottima, per il discorso già affrontato nella sezione \ref{sub:10-prof-it}.

%-----------------------------------------------------------------------------------------------------------
\section{Dominio della metropolitana di Londra}
Questo dominio, affrontato con le tecniche di ricerca in profondità già ampiamente illustrate, non offre particolari nuovi spunti di discussione rispetto ai domini precedenti. Questo, fondamentalmente, è dovuto al fatto che la caratteristica principale che differenzia questo dominio dai precedenti è la presenza di peso sugli archi del grafo (laddove, precedentemente, ogni mossa aveva una valore unitario). La mera ricerca in profondità, non tenendo conto di questo, "appiattisce" questo dominio rendendolo del tutto simile ai precedenti. Sarà più interessante l'analisi all'interno del capitolo \ref{cap:ric-inf}, sulle Ricerche Informate.


\chapter{Ricerche in ampiezza}
In questa parte verranno illustrati gli esercizi inerenti la ricerca in ampiezza sui vari domini.

\section{Dominio dei Cammini (10x10)}\label{sec:10-amp}
\subsection{Ricerca in ampiezza semplice}\label{sec:10-amp-semplice}
Con la ricerca in ampiezza, teoricamente, si dovrebbe trovare una soluzione anche in presenza di cicli infiniti, proprietà non garantita, invece, con la ricerca in profondità. Tuttavia, senza escludere gli stati precedentemente visitati, lo spazio a disposizione dello stack viene esaurito rapidamente, causando un errore di \texttt{Out of global stack}. Questo, come detto, è comprensibile in quanto la ricerca in ampiezza espande ogni cammino possibile, compresi i cicli. Prima del raggiungimento del goal è possibile, quindi, che l'espansione continua di cammini che andrebbero idealmente "potati" (ovvero i cicli), occupi un tale spazio in memoria da compromettere il funzionamento desiderato dell'algoritmo.

\subsection{Ricerca in ampiezza con stati visitati}
Questo raffinamento risolve il problema esposto nella sezione precedente (\ref{sec:10-amp-semplice}), consentendo il raggiungimento della soluzione in un tempo più che accettabile.

%-----------------------------------------------------------------------------------------------------------
\section{Dominio dei Cammini (20x20)} \label{sec:20-amp}
In questo dominio, a differenza di quanto accadeva per la ricerca in profondità, non si vedono sostanziali differenze di tempo con il dominio precedente (\ref{sec:10-amp}) per quanto riguarda la Ricerca in ampiezza con stati visitati\footnote{continua a presentare l'errore \texttt{Out of global stack}, invece, la Ricerca in ampiezza semplice}. Questa è una caratteristica sicuramente positiva, che rende la ricerca in ampiezza preferibile sulla ricerca in profondità per questa tipologia di problemi; oltretutto, la prima soluzione trovata, ha anche la proprietà di essere ottimale.

%-----------------------------------------------------------------------------------------------------------
\section{Dominio del mondo dei blocchi - prima configurazione}
Questa prima configurazione del dominio del mondo dei blocchi non presenta particolari problematiche, trovando le soluzioni in tempo accetabile.

%-----------------------------------------------------------------------------------------------------------
\section{Dominio del mondo dei blocchi - seconda configurazione} \label{sec:2-blocchi-amp}
Questa seconda configurazione risulta, invece, molto più interessante della prima. La maggiore complessità porta a un accrescimento notevole delle configurazioni di stati e, conseguenza di ciò, è il raggiungimento dell'errore \texttt{Out of global stack}. La memoria viene quindi rapidamente esaurita, permettendo di apprezzare la caratteristica principale che si aveva, invece, nell'iterative deepening (come illustrato nella sezione \ref{sub:blocchi-prof-id}) dove, seppur in tempo ampio, veniva trovata una soluzione.

%-----------------------------------------------------------------------------------------------------------
\section{Dominio della metropolitana di Londra} \label{sec:tube-amp}
La ricerca in ampiezza si adatta molto bene a questo dominio che, non presentando una complessità ampia come il dominio \ref{sec:blocchi}, permette il raggiungimento del goal in un tempo particolarmente vantaggioso. Anche in questa sede ci sarebbero da fare le considerazioni fatte nella sezione \ref{sub:tube-prof} sull'appiattimento del dominio.


\chapter{Strategia di ricerca informate} \label{cap:ric-inf}
\section{Euristiche utilizzate}

\subsection{Euristica dei cammini}

\subsection{Euristica del mondo dei blocchi}

\subsection{Euristica della Metropolitana di Londra}

%-----------------------------------------------------------------------------------------------------------
\section{A-star}

\subsection{Cammini}

\subsection{Mondo dei Blocchi}

\subsection{Metropolitana di Londra}
%-----------------------------------------------------------------------------------------------------------
\section{IDA-star}

\subsection{Cammini}

\subsection{Mondo dei Blocchi}

\subsection{Metropolitana di Londra}


\part{CLINGO}
In questa parte verrà descritto lo svolgimento dell'esercizio delle Cinque Case.
Il problema viene così enunciato:
\small{\begin{quote}
    Cinque persone di nazionalità diverse vivono in cinque case allineate lungo una
    strada, esercitano cinque professioni distinte, e ciascuna persona ha un animale favorito e una
    bevanda favorita, tutti diversi fra loro. Le cinque case sono dipinte con colori diversi. Sono noti i
    seguenti fatti:
    \begin{enumerate}
        \item{L’inglese vive nella casa rossa.}
        \item{Lo spagnolo possiede un cane.}
        \item{Il giapponese è un pittore.}
        \item{L’italiano beve tè.}
        \item{Il norvegese vive nella prima casa a sinistra.}
        \item{Il proprietario della casa verde beve caffè.}
        \item{La casa verde è immediatamente sulla destra di quella bianca.}
        \item{Lo scultore alleva lumache.}
        \item{Il diplomatico vive nella casa gialla.}
        \item{Nella casa di mezzo si beve latte.}
        \item{La casa del norvegese è adiacente a quella blu.}
        \item{Il violinista beve succo di frutta.}
        \item{La volpe è nella casa adiacente a quella del dottore.}
        \item{Il cavallo è nella casa adiacente a quella del diplomatico.}
    \end{enumerate}
    Trovare chi possiede una zebra.
\end{quote}}
Il file che formalizza e risolve il problema si chiama \texttt{houses.cl} ed è suddiviso in 6 sezioni.
\paragraph{Dominio}
In questa sezione vengono elencate le entità che figurano nel problema e i valori che queste assumono.
L'entità \texttt{casa} è stata semplicemente codificata con un numero da 1 a 5 (usando il costrutto \emph{Interval} di \texttt{clingo}) poiché le nazionalità dei suoi inquilini, le loro professioni e i loro animali e bevande preferiti sono stati implementati come predicati binari delle istanze di \texttt{casa}. Seguono quindi le entità \texttt{colore}, \texttt{nazionalita}, \texttt{animale}, \texttt{professione} e \texttt{bevanda}. L'uso del carattere \texttt{;} è una sintassi abbreviata per non dover enunciare più volte lo stesso \emph{atomo} con valori diversi. Ad es: \texttt{a(x;y;z)} equivale a: \texttt{a(x). a(y). a(z).}
\clearpage{}
\begin{lstlisting}[frame=tb]
%% Dominio

% Casa
casa(1..5).

% Colore
colore(rossa;verde;bianca;gialla;blu).

% Nazionalita`
nazionalita(inglese;spagnolo;giapponese;italiano;norvegese).

% Animale
animale(cane;lumache;volpe;cavallo;zebra).

% Professione
professione(pittore;scultore;diplomatico;violinista;dottore).

% Bevanda
bevanda(te;caffe;latte;succo_di_frutta;altro).
\end{lstlisting}
\paragraph{Vincoli impliciti}
In questa sezione vengono espressi quelli che sono vincoli "naturali", non espressamente descritti nell'enunciato del problema, ma che è più che ragionevole supporre. Vengono inoltre introdotti i predicati binari \texttt{coloreDi}, \texttt{nazionalitaDi}, \texttt{animaleDi}, \texttt{prefessioneDi} e \texttt{bevandaDi} precedentementi accennati. Ciascuno associa alla casa (passata come primo argomento) una proprietà (secondo argomento) avente un valore apparetenente al relativo dominio.
\begin{lstlisting}[frame=tb]
%% Vincoli impliciti

% Ogni casa ha uno ed un solo colore...
1 {coloreDi(X,C) : colore(C)} 1 :- casa(X).
% ...e due case con lo stesso colore sono la stessa casa
:- casa(X), casa(Y), coloreDi(X,C), coloreDi(Y,C), X != Y.

% In ogni casa abita una persona di una ed una sola nazionalita`...
1 {nazionalitaDi(X,N) : nazionalita(N)} 1 :- casa(X).
% ...e due case con la persona della stessa nazionalita` sono la stessa casa
:- casa(X), casa(Y), nazionalitaDi(X,N), nazionalitaDi(Y,N), X != Y.

% In ogni casa abita una persona che ama uno ed un solo animale...
1 {animaleDi(X,A) : animale(A)} 1 :- casa(X).
% ...e due case con la persona che ama lo stesso animale sono la stessa casa
:- casa(X), casa(Y), animaleDi(X,A), animaleDi(Y,A), X != Y.

% In ogni casa abita una persona che svolge una ed una sola professione...
1 {professioneDi(X,P) : professione(P)} 1 :- casa(X).
% ...e due case con la persona che svolge la stessa preofessione sono la stessa casa
:- casa(X), casa(Y), professioneDi(X,P), professioneDi(Y,P), X != Y.

% In ogni casa abita una persona che predilige una ed una sola bevanda...
1 {bevandaDi(X,B) : bevanda(B)} 1 :- casa(X).
% ...e due case con la persona che predilige la stessa bevanda sono la stessa casa
:- casa(X), casa(Y), bevandaDi(X,B), bevandaDi(Y,B), X != Y.
\end{lstlisting}
\paragraph{Convenzioni e funzione di adiacenza}
Queste due sezioni trattano il posizionamento, assoluto e reciproco, delle case; hanno gli scopi rispettivamente di: \begin{enumerate} \item{esprimere - soltanto attraverso commenti - quello che consideriamo essere ancora una volta una convenzione "a buon senso" circa la disposizione delle case lungo la via} \item{definire una funzione di utilità, \texttt{adj}, che, dati due inter rappresentanti le case, restituisce un valore di verità se queste sono adiacenti, ossia se il valore assoluto della loro differenza è esattamente uguale a 1; questo predicato tornerà utile nella sezione successiva} \end{enumerate}
\begin{lstlisting}[frame=tb]
%% Convenzioni

% casa(1) e` quella posta piu` a sinistra
% casa(5) e` quella posta piu` a destra
% casa(X) e` a destra di casa(Y) <=> X > Y
% casa(X) e` a sinistra di casa(Y) <=> X < Y

%% Funzione di adiacenza

adj(X,Y) :- casa(X), casa(Y), |X-Y|==1.
\end{lstlisting}
\paragraph{Vincoli espliciti}
Questa sezione costituisce il cuore del problema, poiché codifica i 14 vincoli espressi nell'enunciato che restringono man mano i possibili risultati ($5!^5 = 3125$, inizialmente) fino a giungere alla soluzione desiderata. Viene riportato il codice e i relativi commenti. Ogni vincolo è espresso nella forma $$\text{:- } A_1, ..., A_n, B$$ dove le regole $A_1, ..., A_n$ (solitamente $n = 2$) descrivono una o due proprietà di una stessa casa, mentre nella condizione $B$ si \emph{nega} quello che è il vincolo, poiché tutta la riga del vincolo è da considerarsi negata. Ad esempio, il primo vincolo afferma che \emph{non può} esserci una casa in cui abita la persona di nazionalità inglese e che abbia il colore $C$, e che questo colore $C$ sia diverso dal valore "rosso". Nei vincoli 11, 13 e 14 compare la funzione di adiacenza descritta in precedenza, anch'essa negata.
\clearpage{}
\begin{lstlisting}[frame=tb]
%% Vincoli espliciti

% 1. L'inglese vive nella casa rossa
:- nazionalitaDi(X,inglese), coloreDi(X,C), C!=rossa.

% 2. Lo spagnolo possiede un cane
:- nazionalitaDi(X,spagnolo), animaleDi(X,A), A!=cane.

% 3. Il giapponese e` un pittore
:- nazionalitaDi(X,giapponese), professioneDi(X,P), P!=pittore.

% 4. L'italiano beve te`
:- nazionalitaDi(X,italiano), bevandaDi(X,B), B!=te.

% 5. Il norvegese vive nella prima casa a sinistra
:- nazionalitaDi(X,norvegese), X!=1.

% 6. Il proprietario della casa verde beve caffe`
:- bevandaDi(X,caffe), coloreDi(X,C), C!=verde.

% 7. La casa verde e` immediatamente sulla destra di quella bianca
:- coloreDi(X,verde), coloreDi(Y,bianca), X!=Y+1.

% 8. Lo scultore alleva lumache
:- professioneDi(X,scultore), animaleDi(X,A), A!=lumache.

% 9.  Il diplomatico vive nella casa gialla
:- professioneDi(X,diplomatico), coloreDi(X,C), C!=gialla.

% 10. Nella casa di mezzo si beve latte
:- bevandaDi(3,B), B!=latte.

% 11. La casa del norvegese e` adiacente a quella blu
:- nazionalitaDi(X,norvegese), coloreDi(Y,blu), not adj(X,Y).

% 12. Il violinista beve succo di frutta
:- professioneDi(X,violinista), bevandaDi(X,B), B!=succo_di_frutta.

% 13. La volpe e` nella casa adiacente a quella del dottore
:- animaleDi(X,volpe), professioneDi(Y,dottore), not adj(X,Y).

% 14. Il cavallo e` nella casa adiacente a quella del diplomatico
:- animaleDi(X,cavallo), professioneDi(Y,diplomatico), not adj(X,Y).
\end{lstlisting}
\paragraph{Output}
Infine, mediante i predicati \texttt{hide\#} e \texttt{show\#}, nascondiamo inizialmente tutti i predicati per poi selettivamente mostrare \texttt{coloreDi}, \texttt{nazionalitaDi}, \texttt{animaleDi}, \texttt{prefessioneDi} e \texttt{bevandaDi}, tutti con \emph{arità} 2. Senza invocare \texttt{hide\#}, avremmo ottenuto anche i predicati del dominio e la funzione di adiacenza.
\begin{lstlisting}[frame=tb]
%% Output

#hide.
#show coloreDi/2.
#show nazionalitaDi/2.
#show animaleDi/2.
#show professioneDi/2.
#show bevandaDi/2.
\end{lstlisting}
L'output prodotto dall'esecuzione del nostro codice, invocando da terminale il comando \texttt{clingo 0 houses.cl}, è il seguente:
\footnotesize{\begin{verbatim}
Answer: 1
coloreDi(5,verde) coloreDi(4,bianca) coloreDi(3,rossa) coloreDi(2,blu)
coloreDi(1,gialla) nazionalitaDi(5,giapponese) nazionalitaDi(4,spagnolo)
nazionalitaDi(3,inglese) nazionalitaDi(2,italiano) nazionalitaDi(1,norvegese)
animaleDi(5,zebra) animaleDi(4,cane) animaleDi(3,lumache) animaleDi(2,cavallo)
animaleDi(1,volpe) professioneDi(5,pittore) professioneDi(4,violinista)
professioneDi(3,scultore) professioneDi(2,dottore) professioneDi(1,diplomatico)
bevandaDi(5,caffe) bevandaDi(4,succo_di_frutta) bevandaDi(3,latte)
bevandaDi(2,te) bevandaDi(1,altro) 
SATISFIABLE

Models      : 1    
Time        : 0.000
  Prepare   : 0.000
  Prepro.   : 0.000
  Solving   : 0.000
\end{verbatim}}
che ci consente di formulare il risultato finale:
\begin{center}
    \begin{tabular}{| l | l | l | l | l | l |} \hline
    \textbf{Casa} & \textbf{Colore} & \textbf{Nazionalità} & \textbf{Animale} & \textbf{Professione} & \textbf{Bevanda}\\ \hline
    1       & gialla    & norvegese     & volpe     & diplomatico   & altro             \\ \hline
    2       & blu       & italiano      & cavallo   & dottore       & te                \\ \hline
    3       & rossa     & inglese       & lumache   & scultore      & latte             \\ \hline
    4       & bianca    & spagnolo      & cane      & violinista    & succo di frutta   \\ \hline
    5       & verde     & giapponese    & zebra     & pittore       & caffe             \\ \hline
    \end{tabular}
\end{center}
E scoprire così che il proprietario della zebra è il pittore giapponese che vive nella casa verde (l'ultima) e beve caffé.

\end{document}

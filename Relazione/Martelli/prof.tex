In questa parte verranno illustrati gli esercizi inerenti la ricerca in profondità sui vari domini.

\section{Dominio dei Cammini (10x10)}

\subsection{Ricerca in profondità semplice} \label{sub:10-prof-semplice}
La prima ricerca che prendiamo in esame è la così detta ricerca semplice, ovvero una ricerca senza controlli di cicli. Come ci si aspetterebbe, a causa dell'assenza di controlli sugli stati già esplorati, una volta lanciato l'interprete prolog con la ricerca semplice viene restituito \texttt{Out of local stack}. Questo è dovuto, in effetti, a una ricursione infinita in quanto i cammini possibili prevedono cicli. L'interprete espande sempre la chiamata ricursiva più "profonda" e, la presenza di cicli nei cammini, non garantisce la terminazione.

\subsection{Ricerca in profondità con controllo di cicli} \label{sub:10-prof-cc}
Per ovviare a questo problema si è implementata la ricerca in profondità con controllo di cicli. L'idea è quella di utilizzare un insieme indicante le posizioni precedentemente visitate, in maniera da "potare" i rami decisionali generanti cicli. L'insieme è chiamato \texttt{visitati} e, a ogni passo di ricorsione, gli viene aggiunta la posizione attuale.

\subsection{Ricerca in profondità limitata} \label{sub:10-prof-limitata}
La ricerca in profondità limitata consiste nel definire un limite di profondità entro il quale la ricerca si deve fermare. Se è possibile dimostrare l'esistenza di un percorso lungo al più quanto il limite fornito viene restituita la sequenza risultato. Viceversa vi sarà un fallimento.
L'implementazione usata per questa strategia consiste nel passaggio del limite a ogni chiamata ricorsiva. Esso viene decrementato di uno al passaggio alla successiva chiamata ricorsiva e, ogni volta, viene controllato che questo limite sia maggiore di zero. Risulta chiaro che una volta arrivati alla base della ricorsione si è a un livello di profondità pari a quello inizialmente definito. Come detto: se il goal viene raggiunto entro il livello stabilito si ha la soluzione, altrimenti, arrivati alla base della ricorsione, si ha un fallimento.

\subsection{Ricerca in profondità ad approfondimento iterativo} \label{sub:10-prof-it}
Per questa versione viene riutilizzata la precedente ricerca (\ref{sub:10-prof-limitata}) impostando il limite inizialmente a 1 (quindi si avrà successo solo nel caso in cui posizione iniziale e finale coincidano). A ogni iterazione viene incrementato il limite di un'unità e nuovamente esplorato il cammino con la la ricerca in profondità limitata. Il principale vantaggio di questo approccio è la sicurezza di trovare una soluzione minima, in quanto i passi hanno tutti ugual costo (non ci sono cammini pesati) e in quanto il procedimento può essere visto (seppure il comportamento sia diverso) in maniera simile a una ricerca in ampiezza. Questo non è del tutto vero: infatti ogni volta si esplorerà in profondità il ramo in questione e, solo successivamente, si esploreranno, sempre in profondità, i nodi di un altro ramo. Tuttavia, considerando il fatto che si esplorerà sempre prima un cammino più corto di uno lungo, la prima soluzione trovata, come per la ricerca in ampiezza, avrà la proprietà desiderabile di essere anche la soluzione minima. In effetti questa strategia combina i vantaggi della ricerca in profondità con quelli della ricerca in ampiezza.

%-----------------------------------------------------------------------------------------------------------
\section{Dominio dei Cammini (20x20)}

\subsection{Ricerca in profondità semplice} \label{sub:20-prof-semplice}
Anche in questo caso, come per il dominio illustrato nella sezione \ref{sec:cammini:10x10}, verrà restituito l'errore \texttt{Out of local stack}. I motivi sono i medesimi illustrati nella sezione \ref{sub:10-prof-semplice}.

\subsection{Ricerca in profondità con controllo di cicli} \label{sub:20-prof-cc}
Come fatto precedentemente, nella sezione \ref{sub:10-prof-cc}, per risolvere il problema \ref{sub:20-prof-semplice}, si è utilizzato un controllo di cicli. Non si ripete il procedimento in questa sede in quanto sufficientemente illustrato nella sezione \ref{sub:10-prof-cc}.

\subsection{Ricerca in profondità limitata} \label{sub:20-prof-limitata}
Sebbene il funzionamento sia del tutto simile a quanto illustrato nella sezione \ref{sub:10-prof-limitata}, la maggiore profondità del dominio implica un numero sensibilmente maggiore delle possibilità di esplorazione in profondità. Questo si ripercuote in maniera piuttosto evidente sui tempi di esecuzione, di gran lunga superiori a quelli del dominio di dimensioni più piccole. È un fatto sicuramente atteso, dovuto all'enorme differenza di numero di possibili percorsi di una data lunghezza rispetto al dominio \ref{sec:cammini:10x10}.

\subsection{Ricerca in profondità ad approfondimento iterativo} \label{sub:20-prof-it}
Le maggiori conseguenze enfatizzate nella sezione \ref{sub:20-prof-limitata} si riscontrano con questa strategia. In effetti, assumendo che la soluzione minima sia lunghezza \texttt{l}, sarà necessario esplorare prima tutti i possibili cammini di tutte le lunghezza che vanno da \texttt{1} a \texttt{l-1} e, constatando quanto ci si possa impiegare a trovare tutti i cammini di una data lunghezza a ogni livello, il tempo di attesa è molto lungo. Le stampe inserite a ogni nuova chiamata con lunghezza maggiore sono indizio di ciò (più il numero della profondità massima aumenta e più, ovviamente, queste stampe si fanno rarefatte).

%-----------------------------------------------------------------------------------------------------------
\section{Dominio del mondo dei blocchi - prima configurazione}

Su questo dominio non ci sono significative differenze rispetto al Dominio dei cammini, conseguentemente non spenderemo ulteriori commenti

%-----------------------------------------------------------------------------------------------------------
\section{Dominio del mondo dei blocchi - seconda configurazione}

Più interessante è la seconda configurazione che, essendo molto più complessa della prima, causa un incremento importante dei tempi necessari a trovare una soluzione. Tralasciando la Ricerca in profondità semplice, che anche qui, a causa di cicli, porta a un errore \texttt{Out of local stack}.

\subsection{Ricerca in profondità con controllo di cicli} \label{sub:blocchi-prof-cc}
La ricerca in profondità con controllo di cicli risulta, a causa della natura del dominio, molto lunga. Il risultato trovato è ben distante da una soluzione ottimale, questo è dovuto al fatto che, in effetti, l'algoritmo si occupa solamente di evitare i possibili cicli, tuttavia, l'insieme delle mosse possibili, molto vasto: conseguentemente la soluzione include diverse mosse non ottimali che inficiano sulle prestazioni temporali.

\subsection{Ricerca in profondità ad approfondimento iterativo}
Anche qui vengono spese ulteriori parole sulla Ricerca in profondità limitata e si passa subito alla Ricerca in profondità ad approfondimento iterativo. I problemi temporali incontrati in \ref{sub:blocchi-prof-cc} si ripercuotono, ovviamente anche qui; tuttavia questo dominio è particolarmente interessante in quanto permette di evidenziare una proprietà fondamentale dell'iterative deepening: nonostante il tempo impiegato sia paragonabile alla soluzione proposta in \ref{sub:blocchi-prof-cc}, con l'approfondimento iterativo viene trovata una soluzione ottima, per il discorso già affrontato nella sezione \ref{sub:10-prof-it}.

%-----------------------------------------------------------------------------------------------------------
\section{Dominio della metropolitana di Londra}
Questo dominio, affrontato con le tecniche di ricerca in profondità già ampiamente illustrate, non offre particolari nuovi spunti di discussione rispetto ai domini precedenti. Questo, fondamentalmente, è dovuto al fatto che la caratteristica principale che differenzia questo dominio dai precedenti è la presenza di peso sugli archi del grafo (laddove, precedentemente, ogni mossa aveva una valore unitario). La mera ricerca in profondità, non tenendo conto di questo, "appiattisce" questo dominio rendendolo del tutto simile ai precedenti. Sarà più interessante l'analisi all'interno del capitolo \ref{cap:ric-inf}, sulle Ricerche Informate.

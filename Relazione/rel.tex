Questa seconda strategia è molto simile alla precedente, perciò non ci inoltreremo in una disamina dettagliata di ogni caso, mentre ci limiteremo a enfatizzare le sole differenze, allo scopo di una lettura più fluida e comprensibile. L'unica differenza, anche se piuttosto corposa, è relativa al modulo che si occupa del calcolo della cella target (\ref{sec:safe-target}).

\section{Modulo: individuazione della cella target} \label{sec:rel-target}
L'individuazione della cella più appetibile nella strategia safe a punteggi relativi è un raffinamento di quella illustrata nel capitolo \ref{cap:base}, sezione \ref{sec:safe-target}. Durante la fase di inizializzazione dell'ambiente viene assegnato un punteggio a ogni cella, in base alla propria tipologia, e salvato nello slot val del template \texttt{score\_cell}. Questo template serve a memorizzare i punteggi delle celle per individuare la più appetibile presente sulla mappa. I valori, che ricordiamo non essere calcolati in questo modulo, bensì nella fase di inizializzazione dell'ambiente, vengono assegnati nel seguente modo: 
\begin{itemize}
	\item Rural: 900\footnote{il valore viene abbassato a -5 una volta informata}
	\item Urban: 1000\footnote{il valore viene abbassato a -5 una volta informata}
	\item Hill: -100
	\item Lake: -5
	\item Border: -100
	\item Gate: -100
\end{itemize}
Il modulo in questione, come prima cosa, si occupa di assegnare, nello slot \texttt{abs\_score} del template \texttt{score\_cell}, un punteggio, per ogni cella, che tenga conto delle tipologie della cella in questione e di quelle limitrofe. Per fare ciò, semplicemente, vengono sommati i valori del campo val delle celle disposte a nord-ovest, nord, nord-est, ovest, est, sud-ovest, sud, sud-est e il campo val della cella in questione, ottenendo un valore che verrà appunto salvato nel campo \texttt{abs\_score}. Viene chiamato \texttt{abs\_score} che sta per "punteggio assoluto", risulterà chiaro in questo contesto, a differenza della precedente strategia, il perché.
In questo modulo ci si occupa di raffinare i punteggi presenti nel campo \texttt{abs\_score} di ogni cella inserendolo nel campo \texttt{rel\_score} della cella stessa. Per ogni cella, infatti, viene calcalolata la distanza di Manhattan (i due cateti del triangolo rettangolo la cui ipotenusa è la distanza "aerea" tra la cella su cui è l'UAV e la cella considerata). Con questa distanza viene calcolato un nuovo punteggio, \texttt{rel\_score} appunto, che consiste nel rapporto tra il valore del campo \texttt{abs\_score} e la distanza di Manhattan.
Questo procedimento serve a dare un grado di interesse alle celle non solo in base alla "densità" di tipologie di celle interessanti nelle celle limitrofe a quella interessata, come avveniva nella strategia illustrata nel capitolo \ref{cap:base}, bensì, il grado di interesse, sarà inversamente proporzionale anche alla distanza della cella dall'UAV. L'idea è di favorire un'esplorazione più intelligente della mappa, cercando di procedere in maniera omogenea limitando il più possibile gli spostamenti da un estremo all'altro della mappa.
Come nella strategia precedente vengono escluse automaticamente le celle di tipo Hill, Border e Gate (a cui, in effetti, non è nemmeno stato assegnato un valore al campo \texttt{abs\_score}): le prime due per impossibilità di muoversi su celle di quel tipo, l'ultima perché non è desiderabile dirigersi verso una cella di questo tipo fino a quando non si è concluso il compito.
Vengono altresì escluse le celle che sono già state contrassegnate come \texttt{invalid-target} nei corrispettivi moduli \ref{sec:safe-uscita} e \ref{sec:safe-tempo} per questa strategia.
Al termine dei calcoli viene scelta la cella con \texttt{rel\_score} maggiore e impostata come target, quindi il focus viene rilasciato asserendo la condizione di controllo \texttt{punteggi\_checked}.
